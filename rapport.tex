\documentclass[a4paper,11pt]{scrartcl}
\usepackage[french]{babel}
\usepackage[utf8]{inputenc}
\usepackage[T1]{fontenc}
\usepackage{pst-all}
\usepackage{vanadin}
\usepackage{isotope}
\usepackage{longtable}


\subject{TP~3 Physique nucléaire}
\title{Perte d'énergie des particules $\alpha$ dans l'air}
\subtitle{Particules $\alpha$ dans l'air}
\author{Mona Dentler, Sabine Engelhardt}
\publishers{Université Joseph Fourier, Grenoble}
\date{\today}

\begin{document}
 \pagestyle{empty}
 \begin{center}
  \makeatletter
   %\titlefont
  \@subject
  \vspace{2cm}

  \Huge
  Perte d'énergie des particules $\alpha$ dans l'air\newline
  \vspace{1cm}
  \Large


  \@author
  \newline
  \@publishers


  \@date
  \makeatother
 \end{center}
 \vfill

 \begin{abstract}
  Le but des cette TP est de connaître la perte d'énergie des particules $\alpha$ dans l'air. Nous avons eu deux sources une source d'Americium 241 (\isotope[241]{Am}) et une source de Plomb 212/ Bismuth 212 (\isotope[212]{Bi}) pour étudier des particules $\alpha$ différentes. L'$\alpha$ perd son énergie par ionisation des atomes de la matière, ici l'air. La perte est proportionnelle au carré de la charge et à la masse de l'$alpha$ et varie beaucoup avec la vitess d'$\alpha$. Plus l'$\alpha$ est lente plus de temps il passe dans l'atome et \c ca augmente la chance d'une interaction. Si l'ionisation est très intense, le trajet de la particule est très court.

  Dans ce TP nous allons étudier le spectre des $\alpha$ émis par les deux sources, calibrer le dispositif éxpermental pour ensuite mesurer le pouvoir d'ionisation des $\alpha$. C'est réalisé par la mesure de la longuer du trajet des $\alpha$ dans l'air.
 \end{abstract}
\newpage
 \pagestyle{scrheadings}
 \tableofcontents
\newpage

 \begin{section}{Etudes des sources}
  \begin{subsection}{La source \isotope[241][95]{Am}}
   La période de l'\isotope[241][95]{Am} est \unit[432,6]{ans} et ce noyeau se désintègre vers le \isotope[237][93]{Np}. A presque \unit[100]{\%} des désintégration sont des désintégration $\alpha$, seule $\unit[4,3\cdot10^{-10}]{\%}$ 
   se fait par fission spontanée.  

   \todo{Schema malen}
   L'energie de la reaction se calcule comme suivante:
    \begin{equation*}
    Q=\left[M\left(\isotope[241]{Am}\right)-M\left(\isotope[237]{Np}\right)-M\left(\isotope[4]{He}\right)\right]c^2\approx\unit[5,638]{MeV}
   \end{equation*}
et pour les desintegrations vers les états excités convenit:
\begin{equation*}
Q^{\ast}=Q-E^{\ast}
  \end{equation*}
avec $E^{\ast}$ l'energie d'etat excitè.\\
La consevation de la quantité de mouvement implique:
\begin{equation*}
m(\alpha)T(\alpha)=m(Np)T(Np)
  \end{equation*}
En outre 
   Les énergies cinétiques des trois $\alpha$ \isotope[4][2]{He} principaux se calculent comme suivante:
   
\todo {ich komme auf 5,58460332 für den grundzustand, was rechnest du?}
   avec $M\left(\isotope[241]{Am}\right)=\unit[241,0568229]{uma},M\left(\isotope[237]{Np}\right)=\unit[237,0481673]{uma}, \\ M\left(\isotope[4]{He}\right)=\unit[4,00266032]{uma}\text{ et }\unit[1]{uma}\cdot c^2=\unit[931,5]{MeV}$.

   Les trois $\alpha$ principaux sont ceux avec la plus grande possibilité d'être émis, ici ce sont les $\alpha$ émis par la désintegration vers \isotope[237]{Np^*} $\nicefrac{5}{2}^-$ avec \unit[84,85]{\%}, la désintegration vers \isotope[237]{Np^*} $\nicefrac{7}{2}^-$ avec \unit[13,23]{\%} et la désintegration vers \isotope[237]{Np^*} $\nicefrac{9}{2}^-$ avec \unit[1,66]{\%}.
   \begin{eqnarray*}
    T_{\nicefrac{5}{2}^-}=T-E^*_{\nicefrac{5}{2}^-}=\unit[5,578]{MeV}\\
    T_{\nicefrac{7}{2}^-}=T-E^*_{\nicefrac{7}{2}^-}=\unit[5,535]{MeV}\\
    T_{\nicefrac{9}{2}^-}=T-E^*_{\nicefrac{7}{2}^-}=\unit[5,479]{MeV}
   \end{eqnarray*}
  \end{subsection}
 
  \begin{subsection}{La source \isotope[212][83]{Bi}}
   Le \isotope[212][83]{Bi} fait partie de la chaîne radioactive du Thorium 232. Car le  \isotope[212]{Bi} n'a qu'une période de \unit[60,5]{min}, la source a été apporté par un technicien pendant la TP. Le [212]{Bi} se désintègre vers le Thalium \isotope[208][81]{Tl} en émettant des particules $\alpha$ \isotope[4][2]{He} et vers le Polonium\isotope[212][84]{Po} par désintégration $\beta^-$ comme le schéma suivant le montre.

   \todo{Schema Bismuth}

   Le Polonium \isotope[212][84]{Po} fait à son tour par \unit[100]{\%} la désintégration $\alpha$ vers le plomb \isotope[208][82]{Pb} avec une période de $T=\unit[298]{ns}$.

   \todo{Schema Pollonium}

   Les trois $\alpha$ principaux sont deux $\alpha$ de la désintégration vers le \isotope[208]{Tl} et l'$\alpha$ de la désintégration du \isotope[212]{Po} avec les énergies suivantes. Pour 100 désintégration du \isotope[212]{Bi} environ 90 particules $\alpha$ sont émis.
   \begin{eqnarray*}
    T_1=\left(M\left(\isotope[212]{Bi}\right)-M\left(\isotope[208]{Pb}\right)-M\left(\isotope[4]{He}\right)\right)c^2\approx\unit[6,154]{MeV}&\text{ avec }\unit[9,7]{\%}\\
    T_2=T_1-E^*_{4^+}=\unit[6,114]{MeV} &\text{ avec }\unit[25,1]{\%}\\
    T_3=\left(M\left(\isotope[212]{Po}\right)-M\left(\isotope[208]{Pb}\right)-M\left(\isotope[4]{He}\right)\right)c^2\approx\unit[8,901]{MeV} &\text{ avec }\unit[55,2]{\%}    
   \end{eqnarray*}
   avec $M\left(\isotope[212]{Bi}\right)=\unit[211,9912715]{uma},M\left(\isotope[208]{Tl}\right)=\unit[207,9820047]{uma}, M\left(\isotope[212]{Po}\right)=\unit[211,9888518]{uma},\newline M\left(\isotope[208]{Pb}\right)=\unit[207,9766359]{uma}, M\left(\isotope[4]{He}\right)=\unit[4,00266032]{uma}\text{ et }\unit[1]{uma}\cdot c^2=\unit[931,5]{MeV}$.
   
   On peut se poser la question pourquoi la désintégration du \isotope[212]{Bi} vers l'état fondamental est plus probable que vers le deuxième état excité parce qu'ils ont tous les deux le même moment angulaire et la même parité $5^+$. C'est assez facile à comprendre car les atomes vont avoir un état énergétiquement favorable, c'est à dire un état stable. Comme l'état fondamental est le plus stable des deux états la désintégration vers l'état fondamental est préféré.

   Le \isotope[212]{Po} a une période très court à cause de la préférence d'un noyeau avec un nombre magique. Le \isotope[208][82]{Pb} est un noyeau double magique avec le nombre magique 84 pour les protons et le nombre magique 126 pour les neutrons. Alors ce noyeau est fortement favorisé du noyeau de \isotope[212]{Po}.
  \end{subsection}

 \end{section}




 \vspace{1cm}
 \begin{flushright}
  \titlefont \textcopyleft\ Mona Dentler et Sabine Engelhardt
 \end{flushright}

\end{document}

